\documentclass[xcolor={dvipsnames}]{beamer}
% This is the file main.tex
%% =============================
\usepackage{ctex} % 增加中文处理

%% =========================================
%% 定义Kotlin语法高亮
\usepackage[dvipsnames]{xcolor}
\usepackage{listings}
%%==============================
\usetheme{CambridgeUS}

\lstdefinelanguage{Kotlin}{
  comment=[l]{//},
  commentstyle={\color{gray}\ttfamily},
  emph={delegate, filter, first, firstOrNull, forEach, lazy, map, mapNotNull, println, return@},
  emphstyle={\color{OrangeRed}},
  identifierstyle=\color{black},
  keywords={abstract, actual, as, as?, break, by, class, companion, continue, data, do, dynamic, else, enum, expect, false, final, for, fun, get, if, import, in, interface, internal, is, null, object, override, package, private, public, return, set, super, suspend, this, throw, true, try, typealias, val, var, vararg, when, where, while},
  keywordstyle={\color{NavyBlue}\bfseries},
  morecomment=[s]{/*}{*/},
  morestring=[b]",
  morestring=[s]{"""*}{*"""},
  ndkeywords={@Deprecated, @JvmField, @JvmName, @JvmOverloads, @JvmStatic, @JvmSynthetic, Array, Byte, Double, Float, Int, Integer, Iterable, Long, Runnable, Short, String},
  ndkeywordstyle={\color{BurntOrange}\bfseries},
  sensitive=true,
  stringstyle={\color{ForestGreen}\ttfamily},
showstringspaces=false,
}
%% ======================================================

\lstset{
  showstringspaces = false,
  basicstyle=\small
}

\title{Kotlin}
\author{刘海宽}
\date{\today}

\begin{document}

\begin{frame}
  \titlepage
\end{frame}

\section*{目录}
\begin{frame}
  \tableofcontents
\end{frame}

\section{介绍}
\subsection{Kotlin 小史}
\subsection{Kotlin 特点}
\begin{frame}{Kotlin 小史}
  \begin{enumerate}
  \item<1-> 2011年7月,JetBrains推出Kotlin项目
  \item<2-> 2012年2月,Kotlin开源
  \item<3-> 2016年2月,Kotlin v1.0发布
  \item<4-> 2017年10月,Google宣布Kotlin成为Android官方开发语言
  \item<5-> 2017年11月,Kotlin v1.2发布
  \item<6-> 2018年10月,Kotlin v1.3发布
  \end{enumerate}
\end{frame}

\begin{frame}{Kotlin 特点}
  \begin{enumerate}
  \item<1-> 语法简单
  \item<1-> 空指针安全
  \item<1-> 语法简单
  \item<1-> 可以编译为JVM字节码或JS和Naive
  \item<1-> 可以使用现有的Java框架和库
  \item<1-> Intellij可以把Java自动转为Kotlin
  \end{enumerate}
\end{frame}

\section{语法}
\begin{frame}[fragile]{Hello World}
  \alert{Kotlin}
    \begin{lstinputlisting}[language=Kotlin]{code/hello_world.kt}
    \end{lstinputlisting}

  \alert{Java}
    \begin{lstinputlisting}[language=java]{code/hello_world.java}
    \end{lstinputlisting}
\end{frame}

\begin{frame}[fragile]{常量与变量}
  \begin{block}{常量}

    使用\emph{val}(即value)关键字定义常量,定义后其值无法修改
    \begin{lstinputlisting}[language=Kotlin]{code/val.kt}
    \end{lstinputlisting}
  \end{block}

  \begin{block}{变量}

  使用\emph{var}(即variable)关键字定义变量
  \begin{lstinputlisting}[language=Kotlin]{code/var.kt}
  \end{lstinputlisting}
  \end{block}
\end{frame}

\begin{frame}{定义函数}
  使用\emph{fun}关键字定义一个函数
  \lstinputlisting[language=Kotlin]{code/fun.kt}
\end{frame}

\begin{frame}{Main进入点}
  Kotlin的main函数不依赖class,是一个\emph{top-level}函数
  \lstinputlisting[language=Kotlin]{code/main.kt}
\end{frame}

\begin{frame}{for}
  \lstinputlisting[language=Kotlin]{code/for.kt}
\end{frame}

\begin{frame}{扩展函数}
  Kotlin的main函数不依赖class,是一个\emph{top-level}函数
  \lstinputlisting[language=Kotlin]{code/string_last_char_ext.kt}
\end{frame}

\begin{frame}{扩展函数}
  \alert{Java}
  \lstinputlisting[language=Java]{code/toast_origin.kt}
  \alert{Kotlin}
  \lstinputlisting[language=Kotlin]{code/toast_ext.kt}
\end{frame}

\begin{frame}{Java和Kotlin互相调用}
  \lstinputlisting[language=Kotlin]{code/user_java.kt}
\end{frame}

\begin{frame}{Java和Kotlin互相调用}
  \lstinputlisting[language=Kotlin]{code/user_kotlin.kt}
\end{frame}

\begin{frame}{NULL检查机制}
  \lstinputlisting[language=Kotlin]{code/null.kt}
\end{frame}

\begin{frame}{NULL检查机制}
  \lstinputlisting[language=Kotlin]{code/check_null.kt}
\end{frame}

\section{交互式命令行工具}
\begin{frame}
  \lstinputlisting[language=Kotlin]{code/kotlinc.sh}
\end{frame}

\section{应用} % 展示应用场景,增长率等
\begin{frame}
\end{frame}

\section{如何在现有项目使用Kotlin}

\end{document}
