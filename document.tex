\documentclass{beamer}
\mode<presentation>
\title{There Is No Largest Prime Number}
\author[Euclid]{Euclid of Alexandria \texttt{euclid@alexandria.edu}}
\date[ISPN '80]{27th International Symposium of Prime Numbers}

\begin{document}

\begin{frame}
  \titlepage
\end{frame}

\begin{frame}
  \frametitle{Outline}
  \tableofcontents
\end{frame}

\begin{frame} 
  \frametitle{What Are Prime Numbers?}
  \begin{definition}
    A \alert{prime number} is a number that has exactly two divisors.
  \end{definition}
  \begin{example}
    \begin{itemize}
      \item 2 is prime (two divisors: 1 and 2).  \pause
      \item 3 is prime (two divisors: 1 and 3).  \pause
      \item 4 is not prime (\alert{three} divisors: 1, 2, and 4).
    \end{itemize}
  \end{example}

\end{frame}

\begin{frame}
  \frametitle{There Is No Largest Prime Number} \framesubtitle{The
    proof uses \textit{reductio ad absurdum}.}

  \begin{theorem}
    There is no largest prime number.
  \end{theorem}
  \begin{proof}
    \begin{enumerate}
      \item<1-> Suppose $p$ were the largest prime number.
      \item<2-> Let $q$ be the product of the first $p$ numbers.
      \item<3-> Then $q + 1$ is not divisible by any of them.
      \item<1-> But $q + 1$ is greater than $1$, thus divisible by
      some prime number not in the first $p$ numbers.\qedhere
    \end{enumerate}
  \end{proof}
  \uncover<4->{The proof used \textit{reductio ad absurdum}.}
\end{frame}

\begin{frame}
  \frametitle{What's Still To Do?}
  \begin{columns}[t]
    \column{.5\textwidth}
    \begin{block}{Answered Questions}
      How many primes are there?
    \end{block}

    \pause

    \column{0.5\textwidth}
    \begin{block}{Open Questions}
      Is every even number the sum of two primes?
      \cite{Goldbach1742}
    \end{block}
  \end{columns}
\end{frame}

\begin{frame}[fragile]
  \frametitle{An Algorithm For Finding Prime Numbers.}

\begin{verbatim}
int main (void)
{
  std::vector<bool> is_prime(100, true);
  for (int i = 2; i < 100; i++)
    if (is_prime[i])
      {
        std::cout << i << " ";
        for (int j = i; j < 100; is_prime[j] = false, j+=i);
      }
  return 0;
}
\end{verbatim}
  \begin{uncoverenv}<2> Note the use of \verb|std::|.
  \end{uncoverenv}
\end{frame}

\begin{thebibliography}{10}
  \bibitem{Goldbach1742}[Goldbach, 1742]
  Christian Goldbach.
  \newblock A problem we should try to solve before the ISPN '43
  deadline,
  \newblock \emph{Letter to Leonhard Euler}, 1742.
\end{thebibliography}
\end{document}
